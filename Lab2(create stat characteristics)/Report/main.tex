\documentclass[a4paper]{article}

\usepackage{mathtools} % математические формулы
\usepackage[T1,T2A]{fontenc} % кириллица
\usepackage[utf8]{inputenc} % кодировка шрифта кириллицы
\usepackage{indentfirst} %делать отступ в начале параграфа
\usepackage{enumerate} % нумерация списков
\usepackage{tabularx} % таблицы
\usepackage[english,russian]{babel} % вставка стороннего текста
\usepackage[12pt]{extsizes}
\usepackage{amsthm, amssymb, amsmath, amsfonts, nccmath, empheq}
\usepackage{color,colortbl} 
\usepackage[warn]{mathtext}
\usepackage{tocloft}  % для отточий в оглавлении
\linespread{1.5}
\usepackage{setspace} % для пробелов между линий
\usepackage{cmap}


\onehalfspacing
\usepackage{float}
\usepackage{graphicx}
\graphicspath{{pictures/}}
\DeclareGraphicsExtensions{.pdf,.png,.jpg}
\usepackage[left=25mm,right=25mm,
    top=2cm,bottom=30mm,bindingoffset=0cm]{geometry}
\renewcommand{\cftsecleader}{\cftdotfill{\cftdotsep}}

\usepackage{hyperref} % для гиперссылки на гитхаб

\author{Тырыкин Я. А. }
\date{March 2021}

\begin{document}
\begin{titlepage}
    \begin{center}
        \mbox{\normalsize{Санкт-Петербургский Политехнический Университет имени Петра Великого}}\\
        \normalsize{Институт Прикладной Математики и Механики}\\
        \large{\textbf{Кафедра "Прикладная Математика"}}
        
        \vfill
        
        \textbf{\Large{Отчет по лабораторной работе №2}}\\
        \textbf{\large{по дисциплине}}\\
        \textbf{\large"Математическая Статистика"}
        
        \vfill
        \raggedleft{Выполнил студент:}\\
        \raggedleft{Тырыкин Я. А.}\\
        \raggedleft{группа 3630102/80401}\\
        \raggedleft{Проверил:}\\
        \raggedleft{к.ф.-м.н., доцент}\\
        \raggedleft{Баженов А. Н.}\\
        
        % \normalsize{
        %     \begin{spacing}{0.5}
        %         \begin{tabular}{cc}
        %         Проверил: & \\\\ 
        %         к.ф.-м.н., доцент & Александр Николаевич Баженов \\\\
        %         Выполнил студент & \\\\
        %         группы 3630102/80401 & Ярослав Алексеевич Тырыкин \\\\
        %          \\\\
        %         \end{tabular}
        %     \end{spacing}
        % }\\
    
        
        \vfill
    
    \end{center}
    
    \begin{center} 
        Санкт-Петербург \\
        2021 
    \end{center}
\end{titlepage}
\newpage

% страница с оглавлением
% \renewcommand{\contentsname}{Оглавление} % можно поменять название оглавление
\begin{center}
    \tableofcontents
\end{center}
\setcounter{page}{2}
\newpage

% страница со сиском графиков
\begin{center}
    \listoftables
\end{center}
\newpage

\section{Постановка задачи}
 Для 5 распределений:
 \begin{enumerate}
    \begin{item}
            Нормальное распределение \textit{N}(\normalsize{\textit{x}}, \normalsize{0}, \normalsize{1})
        \end{item}
        \begin{item}
            Распределение Коши \textit{C}(\normalsize{\textit{x}}, \normalsize{0}, \normalsize{1})
        \end{item}
        \begin{item}
            Распределение Лапласа \textit{L}(\normalsize{\textit{x}}, \normalsize{0}, \scriptsize{\dfrac{1}{\sqrt{2}}})
        \end{item} 
        \begin{item} 
            Распределение Пуассона \textit{P}(\normalsize{\textit{k}}, \normalsize{10})
        \end{item}
        \begin{item}
            Равномерное распределение \textit{U}(\normalsize{\textit{x}}, \normalsize{$-$\sqrt{3}}, \normalsize{\sqrt{3}})
        \end{item}
 \end{enumerate}
 Сгенерировать выборки размером 10, 100 и 1000 элементов.
Для каждой выборки вычислить следующие статистические характеристики положения данных: $\overline{x}$, \textit{med} $x$, $z_R$, $z_Q$, $z_{tr}$ . Повторить такие
вычисления 1000 раз для каждой выборки и найти среднее характеристик положения и их квадратов:
\begin{equation} \label{E}
    E(z) = \overline{z}
\end{equation}
Вычислить оценку дисперсии по формуле:
    \begin{equation}\label{D}
        D(z) = \overline{z^2} - {\overline{z}}^2
    \end{equation}
Представить полученные данные в виде таблиц.
\section{Теория}
    \subsection{Распределения}
    \begin{itemize}
        \begin{item}
            Нормальное распределение:
        \end{item}
        \begin{equation}\label{norm}
            \textit{N}(\normalsize{\textit{x}}, \normalsize{0}, \normalsize{1}) = \dfrac{1}{\sqrt{2\pi}}e^{\frac{{-x}^2}{2}}
        \end{equation}
        
        \begin{item}
            Распределение Коши:
        \end{item}
        \begin{equation}\label{cauc}
            \textit{C}(\normalsize{\textit{x}}, \normalsize{0}, \normalsize{1}) = \dfrac{1}{\pi}\dfrac{1}{x^2+1}
        \end{equation} 
        
        \begin{item}
            Распределение Лапласа:
        \end{item}
        \begin{equation}\label{lapl}
            \textit{L}(\normalsize{\textit{x}}, \normalsize{0}, \normalsize{\dfrac{1}{\sqrt{2}}}) = \normalsize{\dfrac{1}{\sqrt{2}}e^{-\sqrt{2}|x|}}
        \end{equation}
        
        \begin{item}
            Распределение Пуассона:
        \end{item}
        \begin{equation}\label{pois}
            \textit{P}(\normalsize{\textit{k}}, \normalsize{10}) = \dfrac{10^\textit{k}}{\textit{k}!}e^{-10}
        \end{equation}
        
        \begin{item}
            Равномерное распределение:
        \end{item}
        \begin{equation}\label{unif}
            \textit{U}(\normalsize{\textit{x}}, \normalsize{-\sqrt{3}}, \normalsize{\sqrt{3}}) = \begin{cases}
                                            \dfrac{1}{2\sqrt{3}} & \text{при $|\textit{x}|\leq\sqrt{3}$}\\
                                            0 & \text{при $|\textit{x}| > \sqrt{3}$}\\
                                       \end{cases}
        \end{equation}
    \end{itemize}
    \subsection{Выборочные числовые характеристики}
        \subsubsection{Характеристики положения}
            \begin{itemize}
                \begin{item}
                    Выборочное среднее:
                \end{item}
                \begin{equation}\label{mean}
                    \overline{\textit{x}} = \dfrac{1}{\textit{n}}\sum\limits_{i=1}^n x_i
                \end{equation}
                
                \begin{item}
                    Выборочная медиана:
                \end{item}
                \begin{equation}\label{med}
                    \textit{med x} = \begin{cases}
                                            x_{(l + 1)} & \text{при $n = 2l + 1$}\\
                                            \dfrac{x_{(l)} + x_{(l + 1)}}{2} & \text{при $n = 2l$}\\
                                       \end{cases}
                \end{equation}
                
                \begin{item}
                    Полусумма экстремальных выборочных элементов:
                \end{item}
                \begin{equation}\label{extr}
                    \textit{z}_R = \dfrac{x_{(1)} + x_{(n)}}{2}
                \end{equation}
                
                \begin{item}
                    Полусумма квартилей:\\
                    Выборочная квартиль $z_p$ порядка \textit{p} определяется формулой
                    \begin{equation}
                        z_p = \begin{cases}
                                            x_{([np] + 1)} & \text{при $np$ дробном,}\\
                                            x_{([np])} & \text{при $np$ целом.}\\
                                       \end{cases}
                    \end{equation}
                    Полусумма квартилей
                    
                \end{item}
                \begin{equation}\label{quart}
                    \textit{z}_Q = \dfrac{x_{1/4} + x_{3/4}}{2}
                \end{equation}
                
                \begin{item}
                    Усечённое среднее:
                \end{item}
                \begin{equation}\label{trunc}
                    \textit{z}_{\textit{tr}} = \dfrac{1}{\textit{n} - 2\textit{r}}\sum\limits_{i=r + 1}^{n-r} x_{(i)}, r \approx \dfrac{n}{4}
                \end{equation}
        
            \end{itemize}
        \subsubsection{Характеристики рассеяния}
            Выборочная дисперсия
            \begin{equation}
                D(x) = \dfrac{1}{\textit{n}}\sum\limits_{i=1}^n {(x_i - \overline{x})}^2
            \end{equation}
            
        
\section{Модульная структура программы}
Лабораторная работа выполнена с применением средств языка Python версии 3.7 в среде разработки PyCharm IDE (в частности, с применением встроенных методов библиотеки SciPy и Numpy). Исходной код лабораторной работы находится по ссылке в приложении к отчёту.
\section{Результаты}
    \subsection{Характеристики положения и рассеяния}
    \textit{Как было проведено округление}:
    
В оценке $x = E \pm D$ вариации подлежит первая цифра после точки.\\
\indent В данном случае $x = 0.0 \pm 0.1k$,\\
\indent k − зависит от доверительной вероятности и вида распределения (рассматри\newline{вается в дальнейшем цикле лабораторных работ)}\\
\indent Округление сделано для $k = 1$        
    
    \begin{table}[H]
        \centering
        \begin{tabular}{|c|c|c|c|c|c|}
            \hline
             normal n = 10 & & & & & \\ \hline
             & \overline{x} \eqref{mean} & \textit{med x} \eqref{med} & \textit{z}_R \eqref{extr} & \textit{z}_Q \eqref{quart} & \textit{z}_{tr} \eqref{trunc}\\ \hline
             $E(z)$ \eqref{E} & 0.0004 & 0.2475 & -0.0034 & 0.3104 & 0.422\\ \hline
             $D(z)$ \eqref{D} & 0.1005 & 0.1319 & 0.1828 & 0.1266 & 0.1929\\ \hline
             $E - \sqrt{D}$ & -0.3189 & -0.1405 & -0.4405 & -0.0414 & -0.0238 \\ \hline
             $E + \sqrt{D}$ & 0.2976 & 0.61 & 0.4235 & 0.6499 & 0.84 \\ \hline
             Estimation & 0.0 & 0.0 & 0.0 & $0^{+1}$ & $0^{+1}$ \\ \hline
             normal n = 100 & & & & & \\ \hline
             & \overline{x} & \textit{med x} & \textit{z}_R & \textit{z}_Q & \textit{z}_{tr}\\ \hline
             $E(z)$ \eqref{E} & 0.0011 & 0.0224 & 0.0181 & 0.0145 & 0.6287\\ \hline
             $D(z)$ \eqref{D} & 0.0093 & 0.0147 & 0.093 & 0.0119 & 0.023 \\ \hline
             $E - \sqrt{D}$ & -0.1066 & -0.1037 & -0.3262 & -0.1038 & 0.4613 \\ \hline
             $E + \sqrt{D}$ & 0.0956 & 0.1454 & 0.3184 & 0.1228 & 0.7845 \\ \hline
             Estimation & 0.0 & 0.0 & 0.0 & 0.0 & $0^{+1}$ \\ \hline
             normal n = 1000 & & & & & \\ \hline
             & \overline{x} & \textit{med x} & \textit{z}_R & \textit{z}_Q & \textit{z}_{tr}\\ \hline
             $E(z)$ \eqref{E} & 0.0007 & 0.0031 & 0.0063 & 0.0022 & 0.6367\\ \hline
             $D(z)$ \eqref{D} & 0.0011 & 0.0016 & 0.0657 & 0.0014 & 0.0026 \\ \hline
             $E - \sqrt{D}$ & -0.0328 & -0.0391 & -0.2587 & -0.0348 & 0.5826 \\ \hline
             $E + \sqrt{D}$ & 0.0321 & 0.0423 & 0.2433 & 0.0372 & 0.6849 \\ \hline
             Estimation & 0.0 & 0.0 & 0.0 & 0.0 & $0^{1}$ \\ \hline
        \end{tabular}
        \caption{Характеристики нормального распределения \eqref{norm}}
        \label{tab:norm_tab}
    \end{table}
    
    \begin{table}[H]
        \centering
        \begin{tabular}{|c|c|c|c|c|c|}
            \hline
             cauchy n = 10 & & & & & \\ \hline
             & \overline{x} \eqref{mean} & \textit{med x} \eqref{med} & \textit{z}_R \eqref{extr} & \textit{z}_Q \eqref{quart} & \textit{z}_{tr} \eqref{trunc}\\ \hline
             $E(z)$ \eqref{E} & 2.0874 & 0.4028 & 10.6049 & 1.0676 & 7.0882\\ \hline
             $D(z)$ \eqref{D} & 1734.692 & 0.4305 & 43204.7097 & 4.703 & 4606.8807\\ \hline
             $E - \sqrt{D}$ & -39.5622 & -0.2534 & -197.2525 & -1.101 & -60.7858 \\ \hline
             $E + \sqrt{D}$ & 43.737 & 1.0589 & 218.4624 & 3.2363 & 74.9622 \\ \hline
             Estimation & - & $0^{+1}_{-1}$ & - & $1^{+2}_{-1}$ & -\\ \hline
             cauchy n = 100 & & & & & \\ \hline
             & \overline{x} & \textit{med x} & \textit{z}_R & \textit{z}_Q & \textit{z}_{tr}\\ \hline
             $E(z)$ \eqref{E} & -1.9491 & 0.027 & -96.1502 & 0.0249 & 7.6371\\ \hline
             $D(z)$ \eqref{D} & 3963.3208 & 0.0247 & 9781880.7879 & 0.0556 & 641.2068 \\ \hline
             $E - \sqrt{D}$ & -64.904 & -0.1302 & -3223.7501 & -0.2109 & -17.6849 \\ \hline
             $E + \sqrt{D}$ & 61.0058 & 0.1841 & 3031.4496 & 0.2607 & 32.9592 \\ \hline
             Estimation & - & 0.0 & - & 0.0 & - \\ \hline
             cauchy n = 1000 & & & & & \\ \hline
             & \overline{x} & \textit{med x} & \textit{z}_R & \textit{z}_Q & \textit{z}_{tr}\\ \hline
             $E(z)$ \eqref{E} & -2.557 & 0.0023 & -1304.1555 & 0.0033 & 7.3359\\ \hline
             $D(z)$ \eqref{D} & 1585.6463 & 0.0024 & 390457865.8927 & 0.0048 & 280.3181 \\ \hline
             $E - \sqrt{D}$ & -42.3772 & -0.0467 & -21064.1622 & -0.0663 & -9.4068 \\ \hline
             $E + \sqrt{D}$ & 37.2632 & 0.0513 & 18455.8513 & 0.0729 & 24.0786 \\ \hline
             Estimation & - & 0.0 & - & 0.0 & - \\ \hline
        \end{tabular}
        \caption{Характеристики распределения Коши \eqref{cauc}}
        \label{tab:cauch_tab}
    \end{table}
    
    \begin{table}[H]
        \centering
        \begin{tabular}{|c|c|c|c|c|c|}
            \hline
             laplace n = 10 & & & & & \\ \hline
             & \overline{x} \eqref{mean} & \textit{med x} \eqref{med} & \textit{z}_R \eqref{extr} & \textit{z}_Q \eqref{quart} & \textit{z}_{tr} \eqref{trunc}\\ \hline
             $E(z)$ \eqref{E} & 0.0112 & 0.1825 & 0.0383 & 0.3002 & 0.4213\\ \hline
             $D(z)$ \eqref{D} & 0.1024 & 0.0843 & 0.4016 & 0.118 & 0.1708\\ \hline
             $E - \sqrt{D}$ & -0.3089 & -0.108 & -0.5955 & -0.0432 & 0.008 \\ \hline
             $E + \sqrt{D}$ & 0.3313 & 0.4729 & 0.672 & 0.6437 & 0.8346 \\ \hline
             Estimation & 0 & 0 & 0 & 0 & $0^{+1}$ \\ \hline
             laplace n = 100 & & & & & \\ \hline
             & \overline{x} & \textit{med x} & \textit{z}_R & \textit{z}_Q & \textit{z}_{tr}\\ \hline
             $E(z)$ \eqref{E} & -0.0038 & 0.0104 & 0.016 & 0.0087 & 0.586\\ \hline
             $D(z)$ \eqref{D} & 0.0103 & 0.0057 & 0.4165 & 0.0099 & 0.0216\\ \hline
             $E - \sqrt{D}$ & -0.1054 & -0.0648 & -0.6294 & -0.091 & 0.4389 \\ \hline
             $E + \sqrt{D}$ & 0.0978 & 0.0856 & 0.6614 & 0.1085 & 0.7331 \\ \hline
             Estimation & 0.0 & 0.0 & $0^{+1}_{-1}$ & 0.0 & $0^{+1}$ \\ \hline
             laplace n = 1000 & & & & & \\ \hline
             & \overline{x} & \textit{med x} & \textit{z}_R & \textit{z}_Q & \textit{z}_{tr}\\ \hline
             $E(z)$ \eqref{E} & -0.0013 & 0.0007 & -0.0087 & 0.0002 & 0.5971\\ \hline
             $D(z)$ \eqref{D} & 0.001 & 0.0005 & 0.3856 & 0.0009 & 0.002\\ \hline
             $E - \sqrt{D}$ & -0.0322 & -0.0218 & -0.6297 & -0.0306 & 0.5526 \\ \hline
             $E + \sqrt{D}$ & 0.0296 & 0.0232 & 0.6123 & 0.0311 & 0.6416 \\ \hline
             Estimation & 0.0 & 0.0 & $0^{+1}_{-1}$ & 0.0 & $0^{+1}_{-1}$ \\ \hline
        \end{tabular}
        \caption{Характеристики распределения Лапласа \eqref{lapl}}
        \label{tab:lapl_tab}
    \end{table}
    
    \begin{table}[H]
        \centering
        \begin{tabular}{|c|c|c|c|c|c|}
            \hline
             pois n = 10 & & & & & \\ \hline
             & \overline{x} \eqref{mean} & \textit{med x} \eqref{med} & \textit{z}_R \eqref{extr} & \textit{z}_Q \eqref{quart} & \textit{z}_{tr} \eqref{trunc}\\ \hline
             $E(z)$ \eqref{E} & 10.0283 & 10.6825 & 10.3165 & 10.9835 & 14.6477\\ \hline
             $D(z)$ \eqref{D} & 0.965 & 1.4219 & 2.0436 & 1.3155 & 1.9667\\ \hline
             $E - \sqrt{D}$ & 9.0459 & 9.49 & 8.887 & 9.8366 & 13.2453 \\ \hline
             $E + \sqrt{D}$ & 11.0107 & 11.875 & 11.746 & 12.1304 & 16.0501 \\ \hline
             Estimation & $10^{+1}_{-1}$ & $10^{+1}_{-1}$ & $10^{+1}_{-1}$ & $10^{+2}_{-1}$ & $14^{+2}_{-1}$ \\ \hline
             pois n = 100 & & & & & \\ \hline
             & \overline{x} & \textit{med x} & \textit{z}_R & \textit{z}_Q & \textit{z}_{tr}\\ \hline
             $E(z)$ \eqref{E} & 9.9989 & 9.909 & 10.8785 & 9.967 & 16.9031\\ \hline
             $D(z)$ \eqref{D} & 0.0997 & 0.2022 & 0.9735 & 0.1529 & 0.2782\\ \hline
             $E - \sqrt{D}$ & 9.6831 & 9.4593 & 9.8918 & 9.576 & 16.3756 \\ \hline
             $E + \sqrt{D}$ & 10.3146 & 10.3587 & 11.8652 & 10.358 & 17.4305 \\ \hline
             Estimation & $9^{+1}_{-1}$ & $9^{+1}_{-1}$ & $9^{+1}_{-1}$ & $9^{+1}_{-1}$ & $16^{+1}_{-1}$ \\ \hline
             pois n = 1000 & & & & & \\ \hline
             & \overline{x} & \textit{med x} & \textit{z}_R & \textit{z}_Q & \textit{z}_{tr}\\ \hline
             $E(z)$ \eqref{E} & 9.9969 & 9.997 & 11.6315 & 9.9925 & 16.9141\\ \hline
             $D(z)$ \eqref{D} & 0.0097 & 0.0025 & 0.6285 & 0.0037 & 0.0253\\ \hline
             $E - \sqrt{D}$ & 9.8983 & 9.9471 & 10.8387 & 9.9317 & 16.7551 \\ \hline
             $E + \sqrt{D}$ & 10.0955 & 10.0469 & 12.4243 & 10.0533 & 17.0732 \\ \hline
             Estimation & $9^{+1}_{-1}$ & $9^{+1}_{-1}$ & $11^{+1}_{-1}$ & $9^{+1}_{-1}$ & $16^{+1}_{-1}$\\ \hline
        \end{tabular}
        \caption{Характеристики распределения Пуассона \eqref{pois}}
        \label{tab:pois_tab}
    \end{table}
    
    \begin{table}[H]
        \centering
        \begin{tabular}{|c|c|c|c|c|c|}
            \hline
             uniform n = 10 & & & & & \\ \hline
             & \overline{x} \eqref{mean} & \textit{med x} \eqref{med} & \textit{z}_R \eqref{extr} & \textit{z}_Q \eqref{quart} & \textit{z}_{tr} \eqref{trunc}\\ \hline
             $E(z)$ \eqref{E} & -0.0016 & 0.3037 & 0.0024 & 0.3178 & 0.4173\\ \hline
             $D(z)$ \eqref{D} & 0.1001 & 0.2241 & 0.0488 & 0.1233 & 0.2134\\ \hline
             $E - \sqrt{D}$ & -0.318 & -0.1697 & -0.2186 & -0.0333 & -0.0447 \\ \hline
             $E + \sqrt{D}$ & 0.3148 & 0.7771 & 0.2234 & 0.6689 & 0.8792 \\ \hline
             Estimation & 0.0 & $0^{+1}$ & 0.0 & 0.0 & $0^{+1}$\\ \hline
             uniform n = 100 & & & & & \\ \hline
             & \overline{x} & \textit{med x} & \textit{z}_R & \textit{z}_Q & \textit{z}_{tr}\\ \hline
             $E(z)$ \eqref{E} & -0.0011 & 0.0328 & 0.0009 & 0.0143 & 0.6392\\ \hline
             $D(z)$ \eqref{D} & 0.0102 & 0.0302 & 0.0006 & 0.0148 & 0.0301\\ \hline
             $E - \sqrt{D}$ & -0.1022 & -0.1409 & -0.0238 & -0.1076 & 0.4657 \\ \hline
             $E + \sqrt{D}$ & 0.0999 & 0.2065 & 0.0257 & 0.1361 & 0.8127 \\ \hline
             Estimation & 0.0 & 0.0 & 0.0 & 0.0 & $0^{+1}$ \\ \hline
             uniform n = 1000 & & & & & \\ \hline
             & \overline{x} & \textit{med x} & \textit{z}_R & \textit{z}_Q & \textit{z}_{tr}\\ \hline
             $E(z)$ \eqref{E} & -0.0007 & 0.0022 & -0.0001 & 0.0017 & 0.6479\\ \hline
             $D(z)$ \eqref{D} & 0.0009 & 0.003 & 0.0 & 0.0014 & 0.0028\\ \hline
             $E - \sqrt{D}$ & -0.0315 & -0.0522 & -0.0026 & -0.0356 & 0.5945 \\ \hline
             $E + \sqrt{D}$ & 0.0301 & 0.0567 & 0.0024 & 0.039 & 0.7012 \\ \hline
             Estimation & 0.0 & 0.0 & 0.0 & 0.0 & $0^{+1}$ \\ \hline
        \end{tabular}
        \caption{Характеристики равномерного распределения \eqref{unif}}
        \label{tab:unif_tab}
    \end{table}
    
    
    
\section{Обсуждение}
    \subsection{Характеристики положения и рассеяния}
        Исходя из данных, приведенных в таблицах, можно судить о том, что дисперсия характеристик рассеяния для распределения Коши является некой
аномалией: значения слишком большие даже при увеличении размера выборки - понятно, что это результат выбросов, которые мы могли наблюдать
в результатах предыдущего задания.

\section{Ресурсы}
    \begin{spacing}{2.5}
        Код программы, реализующей отрисовку обозначенных распределений:
        
        \href{https://github.com/YaroslavAggressive/Mathematical-statistics-lab-works}{https://github.com/YaroslavAggressive/Mathematical-statistics-lab-works}
    \end{spacing}
\end{document}
