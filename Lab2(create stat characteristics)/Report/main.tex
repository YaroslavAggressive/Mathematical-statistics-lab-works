\documentclass[a4paper]{article}

\usepackage{mathtools} % математические формулы
\usepackage[T1,T2A]{fontenc} % кириллица
\usepackage[utf8]{inputenc} % кодировка шрифта кириллицы
\usepackage{indentfirst} %делать отступ в начале параграфа
\usepackage{enumerate} % нумерация списков
\usepackage{tabularx} % таблицы
\usepackage[english,russian]{babel} % вставка стороннего текста
\usepackage[12pt]{extsizes}
\usepackage{amsthm, amssymb, amsmath, amsfonts, nccmath, empheq}
\usepackage{color,colortbl} 
\usepackage[warn]{mathtext}
\usepackage{tocloft}  % для отточий в оглавлении
\linespread{1.5}
\usepackage{setspace} % для пробелов между линий
\usepackage{cmap}


\onehalfspacing
\usepackage{float}
\usepackage{graphicx}
\graphicspath{{pictures/}}
\DeclareGraphicsExtensions{.pdf,.png,.jpg}
\usepackage[left=25mm,right=25mm,
    top=2cm,bottom=30mm,bindingoffset=0cm]{geometry}
\renewcommand{\cftsecleader}{\cftdotfill{\cftdotsep}}

\usepackage{hyperref} % для гиперссылки на гитхаб

\author{Тырыкин Я. А. }
\date{March 2021}

\begin{document}
\begin{titlepage}
    \begin{center}
        \mbox{\normalsize{Санкт-Петербургский Политехнический Университет имени Петра Великого}}\\
        \normalsize{Институт Прикладной Математики и Механики}\\
        \large{\textbf{Кафедра "Прикладная Математика"}}
        
        \vfill
        
        \textbf{\Large{Отчет по лабораторной работе №2}}\\
        \textbf{\large{по дисциплине}}\\
        \textbf{\large"Математическая Статистика"}
        
        \vfill
        \raggedleft{Выполнил студент:}\\
        \raggedleft{Тырыкин Я. А.}\\
        \raggedleft{группа 3630102/80401}\\
        \raggedleft{Проверил:}\\
        \raggedleft{к.ф.-м.н., доцент}\\
        \raggedleft{Баженов А. Н.}\\
        
        % \normalsize{
        %     \begin{spacing}{0.5}
        %         \begin{tabular}{cc}
        %         Проверил: & \\\\ 
        %         к.ф.-м.н., доцент & Александр Николаевич Баженов \\\\
        %         Выполнил студент & \\\\
        %         группы 3630102/80401 & Ярослав Алексеевич Тырыкин \\\\
        %          \\\\
        %         \end{tabular}
        %     \end{spacing}
        % }\\
    
        
        \vfill
    
    \end{center}
    
    \begin{center} 
        Санкт-Петербург \\
        2021 
    \end{center}
\end{titlepage}
\newpage

% страница с оглавлением
% \renewcommand{\contentsname}{Оглавление} % можно поменять название оглавление
\begin{center}
    \tableofcontents
\end{center}
\setcounter{page}{2}
\newpage

% страница со сиском графиков
\begin{center}
    \listoftables
\end{center}
\newpage

\section{Постановка задачи}
 Для 5 распределений:
 \begin{enumerate}
    \begin{item}
            Нормальное распределение \textit{N}(\normalsize{\textit{x}}, \normalsize{0}, \normalsize{1})
        \end{item}
        \begin{item}
            Распределение Коши \textit{C}(\normalsize{\textit{x}}, \normalsize{0}, \normalsize{1})
        \end{item}
        \begin{item}
            Распределение Лапласа \textit{L}(\normalsize{\textit{x}}, \normalsize{0}, \scriptsize{\dfrac{1}{\sqrt{2}}})
        \end{item} 
        \begin{item} 
            Распределение Пуассона \textit{P}(\normalsize{\textit{k}}, \normalsize{10})
        \end{item}
        \begin{item}
            Равномерное распределение \textit{U}(\normalsize{\textit{x}}, \normalsize{$-$\sqrt{3}}, \normalsize{\sqrt{3}})
        \end{item}
 \end{enumerate}
 Сгенерировать выборки размером 10, 100 и 1000 элементов.
Для каждой выборки вычислить следующие статистические характеристики положения данных: $\overline{x}$, \textit{med} $x$, $z_R$, $z_Q$, $z_{tr}$ . Повторить такие
вычисления 1000 раз для каждой выборки и найти среднее характеристик положения и их квадратов:
\begin{equation} \label{E}
    E(z) = \overline{z}
\end{equation}
Вычислить оценку дисперсии по формуле:
    \begin{equation}\label{D}
        D(z) = \overline{z^2} - {\overline{z}}^2
    \end{equation}
Представить полученные данные в виде таблиц.
\section{Теория}
    \subsection{Распределения}
    \begin{itemize}
        \begin{item}
            Нормальное распределение:
        \end{item}
        \begin{equation}\label{norm}
            \textit{N}(\normalsize{\textit{x}}, \normalsize{0}, \normalsize{1}) = \dfrac{1}{\sqrt{2\pi}}e^{\frac{{-x}^2}{2}}
        \end{equation}
        
        \begin{item}
            Распределение Коши:
        \end{item}
        \begin{equation}\label{cauc}
            \textit{C}(\normalsize{\textit{x}}, \normalsize{0}, \normalsize{1}) = \dfrac{1}{\pi}\dfrac{1}{x^2+1}
        \end{equation} 
        
        \begin{item}
            Распределение Лапласа:
        \end{item}
        \begin{equation}\label{lapl}
            \textit{L}(\normalsize{\textit{x}}, \normalsize{0}, \normalsize{\dfrac{1}{\sqrt{2}}}) = \normalsize{\dfrac{1}{\sqrt{2}}e^{-\sqrt{2}|x|}}
        \end{equation}
        
        \begin{item}
            Распределение Пуассона:
        \end{item}
        \begin{equation}\label{pois}
            \textit{P}(\normalsize{\textit{k}}, \normalsize{10}) = \dfrac{10^\textit{k}}{\textit{k}!}e^{-10}
        \end{equation}
        
        \begin{item}
            Равномерное распределение:
        \end{item}
        \begin{equation}\label{unif}
            \textit{U}(\normalsize{\textit{x}}, \normalsize{-\sqrt{3}}, \normalsize{\sqrt{3}}) = \begin{cases}
                                            \dfrac{1}{2\sqrt{3}} & \text{при $|\textit{x}|\leq\sqrt{3}$}\\
                                            0 & \text{при $|\textit{x}| > \sqrt{3}$}\\
                                       \end{cases}
        \end{equation}
    \end{itemize}
    \subsection{Выборочные числовые характеристики}
        \subsubsection{Характеристики положения}
            \begin{itemize}
                \begin{item}
                    Выборочное среднее:
                \end{item}
                \begin{equation}\label{mean}
                    \overline{\textit{x}} = \dfrac{1}{\textit{n}}\sum\limits_{i=1}^n x_i
                \end{equation}
                
                \begin{item}
                    Выборочная медиана:
                \end{item}
                \begin{equation}\label{med}
                    \textit{med x} = \begin{cases}
                                            x_{(l + 1)} & \text{при $n = 2l + 1$}\\
                                            \dfrac{x_{(l)} + x_{(l + 1)}}{2} & \text{при $n = 2l$}\\
                                       \end{cases}
                \end{equation}
                
                \begin{item}
                    Полусумма экстремальных выборочных элементов:
                \end{item}
                \begin{equation}\label{extr}
                    \textit{z}_R = \dfrac{x_{(1)} + x_{(n)}}{2}
                \end{equation}
                
                \begin{item}
                    Полусумма квартилей:\\
                    Выборочная квартиль $z_p$ порядка \textit{p} определяется формулой
                    \begin{equation}
                        z_p = \begin{cases}
                                            x_{([np] + 1)} & \text{при $np$ дробном,}\\
                                            x_{([np])} & \text{при $np$ целом.}\\
                                       \end{cases}
                    \end{equation}
                    Полусумма квартилей
                    
                \end{item}
                \begin{equation}\label{quart}
                    \textit{z}_Q = \dfrac{x_{1/4} + x_{3/4}}{2}
                \end{equation}
                
                \begin{item}
                    Усечённое среднее:
                \end{item}
                \begin{equation}\label{trunc}
                    \textit{z}_{\textit{tr}} = \dfrac{1}{\textit{n} - 2\textit{r}}\sum\limits_{i=r + 1}^{n-r} x_{(i)}, r \approx \dfrac{n}{4}
                \end{equation}
        
            \end{itemize}
        \subsubsection{Характеристики рассеяния}
            Выборочная дисперсия
            \begin{equation}
                D(x) = \dfrac{1}{\textit{n}}\sum\limits_{i=1}^n {(x_i - \overline{x})}^2
            \end{equation}
            
        
\section{Модульная структура программы}
Лабораторная работа выполнена с применением средств языка Python версии 3.7 в среде разработки PyCharm IDE (в частности, с применением встроенных методов библиотеки SciPy и Numpy). Исходной код лабораторной работы находится по ссылке в приложении к отчёту.
\section{Результаты}
    \subsection{Характеристики положения и рассеяния}
    \textit{Как было проведено округление}:
    
В оценке $x = E \pm D$ вариации подлежит первая цифра после точки.\\
\indent В данном случае $x = 0.0 \pm 0.1k$,\\
\indent k − зависит от доверительной вероятности и вида распределения (рассматри\newline{вается в дальнейшем цикле лабораторных работ)}\\
\indent Округление сделано для $k = 1$        
    
    \begin{table}[H]
        \centering
        \begin{tabular}{|c|c|c|c|c|c|}
            \hline
             normal n = 10 & & & & & \\ \hline
             & \overline{x} \eqref{mean} & \textit{med x} \eqref{med} & \textit{z}_R \eqref{extr} & \textit{z}_Q \eqref{quart} & \textit{z}_{tr} \eqref{trunc}\\ \hline
             $E(z)$ \eqref{E} & 0.0004 & 0.2475 & -0.0034 & 0.3104 & 0.422\\ \hline
             $D(z)$ \eqref{D} & 0.1005 & 0.1319 & 0.1828 & 0.1266 & 0.1929\\ \hline
             normal n = 100 & & & & & \\ \hline
             & \overline{x} & \textit{med x} & \textit{z}_R & \textit{z}_Q & \textit{z}_{tr}\\ \hline
             $E(z)$ \eqref{E} & 0.0011 & 0.0224 & 0.0181 & 0.0145 & 0.6287\\ \hline
             $D(z)$ \eqref{D} & 0.0093 & 0.0147 & 0.093 & 0.0119 & 0.023 \\ \hline
             normal n = 1000 & & & & & \\ \hline
             & \overline{x} & \textit{med x} & \textit{z}_R & \textit{z}_Q & \textit{z}_{tr}\\ \hline
             $E(z)$ \eqref{E} & 0.0007 & 0.0031 & 0.0063 & 0.0022 & 0.6367\\ \hline
             $D(z)$ \eqref{D} & 0.0011 & 0.0016 & 0.0657 & 0.0014 & 0.0026 \\ \hline
        \end{tabular}
        \caption{Характеристики нормального распределения \eqref{norm}}
        \label{tab:norm_tab}
    \end{table}
    
    \begin{table}[H]
        \centering
        \begin{tabular}{|c|c|c|c|c|c|}
            \hline
             cauchy n = 10 & & & & & \\ \hline
             & \overline{x} \eqref{mean} & \textit{med x} \eqref{med} & \textit{z}_R \eqref{extr} & \textit{z}_Q \eqref{quart} & \textit{z}_{tr} \eqref{trunc}\\ \hline
             $E(z)$ \eqref{E} & 3.2342 & 0.4268 & 16.1933 & 1.127 & 9.0583\\ \hline
             $D(z)$ \eqref{D} & 3296.7681 & 0.4775 & 82205.7055 & 3.5466 & 9000.5072\\ \hline
             cauchy n = 100 & & & & & \\ \hline
             & \overline{x} & \textit{med x} & \textit{z}_R & \textit{z}_Q & \textit{z}_{tr}\\ \hline
             $E(z)$ \eqref{E} & -2.4426 & 0.0369 & -120.455 & 0.0331, 6.0248\\ \hline
             $D(z)$ \eqref{D} & 5229.3575 & 0.0246 & 13034764.0549 & 0.053 & 437.5538 \\ \hline
             cauchy n = 1000 & & & & & \\ \hline
             & \overline{x} & \textit{med x} & \textit{z}_R & \textit{z}_Q & \textit{z}_{tr}\\ \hline
             $E(z)$ \eqref{E} & -3.7405 & 0.0035 & -1886.7218 & 0.0028 & 32.8812\\ \hline
             $D(z)$ \eqref{D} & 351915.338 & 0.0026 & 87839440642.4975 & 0.0056 & 276417.646 \\ \hline
        \end{tabular}
        \caption{Характеристики распределения Коши \eqref{cauc}}
        \label{tab:cauch_tab}
    \end{table}
    
    \begin{table}[H]
        \centering
        \begin{tabular}{|c|c|c|c|c|c|}
            \hline
             laplace n = 10 & & & & & \\ \hline
             & \overline{x} \eqref{mean} & \textit{med x} \eqref{med} & \textit{z}_R \eqref{extr} & \textit{z}_Q \eqref{quart} & \textit{z}_{tr} \eqref{trunc}\\ \hline
             $E(z)$ \eqref{E} & 0.0197 & 0.1889 & 0.05 & 0.3222 & 0.4359\\ \hline
             $D(z)$ \eqref{D} & 0.0975 & 0.0783 & 0.4115 & 0.1317 & 0.1767\\ \hline
             laplace n = 100 & & & & & \\ \hline
             & \overline{x} & \textit{med x} & \textit{z}_R & \textit{z}_Q & \textit{z}_{tr}\\ \hline
             $E(z)$ \eqref{E} & -0.0067 & 0.0113 & 0.0098 & 0.0047 & 0.5806\\ \hline
             $D(z)$ \eqref{D} & 0.0095 & 0.0054 & 0.4287 & 0.0096 & 0.0203\\ \hline
             laplace n = 1000 & & & & & \\ \hline
             & \overline{x} & \textit{med x} & \textit{z}_R & \textit{z}_Q & \textit{z}_{tr}\\ \hline
             $E(z)$ \eqref{E} & -0.0013 & 0.0008 & -0.0042 & 0.001 & 0.5965\\ \hline
             $D(z)$ \eqref{D} & 0.001 & 0.0006 & 0.394 & 0.0011 & 0.0022\\ \hline
        \end{tabular}
        \caption{Характеристики распределения Лапласа \eqref{lapl}}
        \label{tab:lapl_tab}
    \end{table}
    
    \begin{table}[H]
        \centering
        \begin{tabular}{|c|c|c|c|c|c|}
            \hline
             pois n = 10 & & & & & \\ \hline
             & \overline{x} \eqref{mean} & \textit{med x} \eqref{med} & \textit{z}_R \eqref{extr} & \textit{z}_Q \eqref{quart} & \textit{z}_{tr} \eqref{trunc}\\ \hline
             $E(z)$ \eqref{E} & 9.9856 & 10.6225 & 10.2885 & 10.9575 & 14.585\\ \hline
             $D(z)$ \eqref{D} & 1.0347 & 1.5097 & 1.8355 & 1.4554 & 2.0978\\ \hline
             pois n = 100 & & & & & \\ \hline
             & \overline{x} & \textit{med x} & \textit{z}_R & \textit{z}_Q & \textit{z}_{tr}\\ \hline
             $E(z)$ \eqref{E} & 9.9906 & 9.9025 & 10.913 & 9.946 & 16.8963\\ \hline
             $D(z)$ \eqref{D} & 0.1008 & 0.1932 & 1.0099 & 0.1621 & 0.2889\\ \hline
             pois n = 1000 & & & & & \\ \hline
             & \overline{x} & \textit{med x} & \textit{z}_R & \textit{z}_Q & \textit{z}_{tr}\\ \hline
             $E(z)$ \eqref{E} & 10.0044 & 10.0 & 11.642 & 9.9965 & 16.9263\\ \hline
             $D(z)$ \eqref{D} & 0.0097 & 0.0 & 0.6868 & 0.0017 & 0.0257\\ \hline
        \end{tabular}
        \caption{Характеристики распределения Пуассона \eqref{pois}}
        \label{tab:pois_tab}
    \end{table}
    
    \begin{table}[H]
        \centering
        \begin{tabular}{|c|c|c|c|c|c|}
            \hline
             uniform n = 10 & & & & & \\ \hline
             & \overline{x} \eqref{mean} & \textit{med x} \eqref{med} & \textit{z}_R \eqref{extr} & \textit{z}_Q \eqref{quart} & \textit{z}_{tr} \eqref{trunc}\\ \hline
             $E(z)$ \eqref{E} & -0.0044 & 0.3122 & 0.0004 & 0.3151 & 0.4151\\ \hline
             $D(z)$ \eqref{D} & 0.097 & 0.2125 & 0.0421 & 0.1293 & 0.2143\\ \hline
             uniform n = 100 & & & & & \\ \hline
             & \overline{x} & \textit{med x} & \textit{z}_R & \textit{z}_Q & \textit{z}_{tr}\\ \hline
             $E(z)$ \eqref{E} & 0.0039 & 0.0386 & 0.0005 & 0.0213 & 0.6512\\ \hline
             $D(z)$ \eqref{D} & 0.0109 & 0.0309 & 0.0006 & 0.0162 & 0.0317\\ \hline
             uniform n = 1000 & & & & & \\ \hline
             & \overline{x} & \textit{med x} & \textit{z}_R & \textit{z}_Q & \textit{z}_{tr}\\ \hline
             $E(z)$ \eqref{E} & -0.0007 & 0.0024 & 0.0001 & 0.0015 & 0.6475\\ \hline
             $D(z)$ \eqref{D} & 0.0011 & 0.0031 & 0.0 & 0.0015 & 0.0033\\ \hline
        \end{tabular}
        \caption{Характеристики равномерного распределения \eqref{unif}}
        \label{tab:unif_tab}
    \end{table}
    
\section{Обсуждение}
    \subsection{Характеристики положения и рассеяния}
        Исходя из данных, приведенных в таблицах, можно судить о том, что дисперсия характеристик рассеяния для распределения Коши является некой
аномалией: значения слишком большие даже при увеличении размера выборки - понятно, что это результат выбросов, которые мы могли наблюдать
в результатах предыдущего задания.

\section{Ресурсы}
    \begin{spacing}{2.5}
        Код программы, реализующей отрисовку обозначенных распределений:
        
        \href{https://github.com/YaroslavAggressive/Mathematical-statistics-lab-works}{https://github.com/YaroslavAggressive/Mathematical-statistics-lab-works}
    \end{spacing}
\end{document}
